Solving the Fujita Rackwitz Problem in dimensions $d \in \{10, 20, \ldots, 120\}$ 200 timesfor sample size $J = 5000$  with different values for $N_{ \textup{ obs } }$  (row), two variants of the CBREE method (column) and different averaging methods of the last $N_{ \textup{ obs } }$ probability of failure estimates (color). Other parameters are fixed. Namely, we use the stopping criterion $\Delta_{\text{Target}} = 2$, the stepsize tolerance $\epsilon_{\text{Target}} = 0.5$, the increase control  of $\sigma$ with $\text{Lip}(\sigma) = 1$ and approximate the indicator function with $I_\text{alg}$.