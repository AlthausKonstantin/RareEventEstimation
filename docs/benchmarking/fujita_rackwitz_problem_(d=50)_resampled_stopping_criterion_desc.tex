Solving Fujita Rackwitz Problem (d=50) with the CBREE methods using      different parameters.     We vary the stopping criterion $\Delta_{\text{Target}}$ (color),     the divergence criterion $N_\text{obs}$ (row) and     the method $\mu^N$ (column).     The parameter $\epsilon_{\text{Target}} = 0.5$     and the choice of the indicator approximation $I_\text{alg}$     are fixed.     Furthermore we plot also the performance of the benchmark methods EnKF    (with different importance sampling densities)    and SiS (with different MCMC sampling methods).     We used the sample sizes $J \in \{1000, 2000, \ldots, 6000\}$.     Each marker represents the empirical estimates based the successful portion of $200$ simulations.