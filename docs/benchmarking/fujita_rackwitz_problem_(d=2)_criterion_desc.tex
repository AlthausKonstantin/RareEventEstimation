Solving the Fujita Rackwitz Problem (d=2) with the CBREE method using  different parameters. We vary the stopping criterion $\Delta_{\text{Target}}$ (color) and the length of the observation window $N_\text{obs}$ (row). The parameter $\epsilon_{\text{Target}} = 0.5$ and the choice of the indicator approximation $I_\text{alg}$ are fixed. Furthermore we plot also the performance of the benchmark methods EnKF and SiS. We used the sample sizes $J \in \{1000, 2000, \ldots, 6000\}$. Each marker represents the empirical estimates based the successful portion of $200$ simulations.