Solving the Fujita Rackwitz (d=2) with the CBREE method using  different parameters. We vary the stopping criterion $\Delta_{\text{Target}}$ (color), the divergence criterion $N_\text{obs}$ (row) and the importance sampling density $\mu^N$ (column). The parameter $\epsilon_{\text{Target}} = 0.5$ and the choice of the indicator approximation $I_\text{alg}$ are fixed. Furthermore we plot also the performance of the benchmark methods EnKF(with different importance sampling densities)and SiS (with different MCMC sampling methods). Each marker represents the point estimates based on 200 simulations.